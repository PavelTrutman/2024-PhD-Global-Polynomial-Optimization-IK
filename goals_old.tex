\chapter{[Old] Goals and future work}
Future work and goals of the thesis can be now divided into two main directions.
First, the method for IK problem solving presented in \refcha{ikt} can be improved in many aspects.
Secondly, the methods of polynomial optimization reviewed in \refcha{pol} are general, and therefore can be applied to other problems from robotics or to even different fields of study, e.g.\ computer vision.

\section{Possible improvements of the IK problem solving method}
We identify several possible improvements to the method for global solving IK problems that we would like to further explore.
\begin{itemize}
  \item Since the presented method is not real-time but takes several seconds to find the solution, we would like to combine this global optimal method with local optimization.
  Optimal IK solutions can be precomputed off-line for a discrete set of poses for the given manipulator.
  Then, the IK solution for new query pose can be computed by local optimization methods from the closest precomputed pose.
  Usage of local optimization methods can significantly improve the query time of the method while still providing near-optimal solutions.
  An evaluation of how far are these non-global optimal solutions from the global ones will reveal how dense the working space of the manipulator has to be sampled in the off-line phase.

  \item The objective function \refeqb{IKT:objectiveSOS} introduced in this text has been chosen arbitrarily.
  Any other objective function can be used in the presented framework as long as it can be expressed as a low degree polynomial in sines and cosines of the joint angles.
  We would like to explore which objective functions are in interest to the robotics community.
  One of the possible candidates is to maximize the dexterity of the manipulator.

  \item The symbolic preprocessing step as described in \refsec{ikt:sym} is currently done for each pose from scratch by a general algorithm for Gr\"obner bases computation.
  As presented in \refsec{pol:autogen}, the runtime of this step can be significantly improved if a specialized solver is used.

  \item Our presented method without the symbolic precomputation step works in general for manipulators with any number of DOFs.
  Therefore, it is natural to explore how it will behave for manipulators with more than 7 DOFs.
  For example, for manipulators with 8 revolute joints, the polynomials from the forward kinematics can be still manipulated in a way that the original polynomial optimization problem will contain polynomials of degree at most four.
  Whether the ideal generated by these polynomials can be still generated just by degree two polynomials is not clear and has to be evaluated.

  \item In path planning, it is important that the IK method generates a smooth path in the joint space if the path in the 3D space is smooth.
  For now, it is not clear to us if this condition is satisfied for our presented method.
  Therefore, experiments that would verify it are in place. 
\end{itemize}

\section{Polynomial optimization in the geometry of computer vision}
Many problems are modeled as polynomial systems in the geometry of computer vision.
Many of them are minimal problems, i.e.\ problems that have a finite number of solutions.
Quite a lot of these minimal problems have been successfully solved by the Gr\"obner bases computation methods.
Moreover, since automatic generators~\cite{AutoGen, Larsson2017} have been proposed many of these problems can be now solved in milliseconds.

Advantage of the Lasserre's hierarchies is that we can limit ourselves to real numbers only, and therefore save some computation time by not computing non-real solutions (and there may be plenty of them), in which we are not interested.

Moreover, we can also add polynomial inequalities to the polynomial systems and reduce the number of superfluous solutions even more. A typical example from computer vision may be to limit the estimated focal lengths to some small reasonable interval.

What methods based on Gr\"obner bases computation can not solve are overconstrained polynomial systems.
But with the usage of polynomial optimization methods, we can relax the constraints of the given problem and the errors of them can be minimized. For example, 4P3V minimal problem~\cite{4p3v} is an overconstrained problem.
