\section{Polynomial algebra preliminaries}
To make this chapter self-contained, we introduce the minimal possible amount of notation for polynomial algebra.
The overall review of the key concepts can be seen in \cite{Cox-Little-Shea2015}, which notation we will follow.

\subsection{The polynomial ring, ideals and varieties}
In this section, we will consider polynomials in $n$ unknowns $x_1, x_2, \ldots, x_n$ with real coefficients.

\begin{definition}[Monomial {\thecite[Chapter~1, §1, Definition~1]{Cox-Little-Shea2015}}]
  A monomial in variables $x_1, x_2, \ldots , x_n$ is a product of the form
  \begin{align}
    x_1^{\alpha_1} x_2^{\alpha_2} \cdots x_n^{\alpha_n},
  \end{align}
  where all of the exponents $\alpha_1, \alpha_2, \ldots, \alpha_n \in \Zpn$.
  The total degree of this monomial is the sum $\alpha_1 + \alpha_2 + \cdots + \alpha_n$.
\end{definition}

When we let $\alpha = \begin{pmatrix}\alpha_1 & \alpha_2 & \cdots & \alpha_n\end{pmatrix}$ to be an $N$-tuple then we can simplify the notation for monomials by setting
\begin{align}
  x^\alpha &= x_1^{\alpha_1} x_2^{\alpha_2} \cdots x_n^{\alpha_n}.
\end{align}
We call the $N$-tuple $\alpha$ the multidegree of the monomial $x^\alpha$.
Let us also denote $|\alpha| = \alpha_1 + \alpha_2 + \cdots + \alpha_n$ the total degree of the monomial $x^\alpha$.
Note that $x^\alpha = 1$ when $\alpha = \begin{pmatrix}0 & 0 & \cdots & 0\end{pmatrix}$.

\begin{definition}[Polynomial {\thecite[Chapter~1, §1, Definition~2]{Cox-Little-Shea2015}}]
  A polynomial $f$ in variables $x_1, x_2, \cdots, x_n$ with real coefficients is a finite linear combination (with real coefficients) of monomials.
  We will write a polynomial $f$ in the form
  \begin{align}
    f &= \sum_{\alpha\in\Zpn^n} c_\alpha x^\alpha,
  \end{align}
where $c_\alpha\in\R$, and there are only finitely many terms $c_\alpha x^\alpha$ in the sum.
\end{definition}

We call $c_\alpha$ the coefficient of the monomial $x^\alpha$.
If $c_\alpha \neq 0$, then we call $c_\alpha x^\alpha$ a term of the polynomial $f$.
The total degree of the polynomial $f$ is denoted as $\deg(f) = \max_{\alpha\in\Zpn^n} |\alpha|$ for non-zero coefficients $c_\alpha\in\R$.
The zero polynomial has no degree.
The set of all polynomials in variables $x = \begin{bmatrix}x_1 & x_2 & \cdots & x_n\end{bmatrix}^\top$ with real coefficients is the ring of multivariate polynomials denoted as $\R[x]$.

\begin{example}
  Let us have polynomial $f \in\R[x, y, z]$ in three variables:
  \begin{align}
    f &= x^3y - 4xz^2 + \frac{2}{3}z^2 + 1.
  \end{align}
  The polynomial $f$ is a sum of four terms.
  The second term $-4xz^2$ with multidegree $\alpha = \begin{pmatrix}1 & 0 & 2\end{pmatrix}$ has a coefficient $-4$ and monomial $xz^2$.
  The total degree of this monomial is $1 + 0 + 2 = 3$, but the total degree $\deg(f)$ of the polynomial $f$ is 4, because that is the multidegree of the monomial $x^3y$.
\end{example}

\begin{definition}[Ideal {\thecite[Chapter~1, §4, Definition~1]{Cox-Little-Shea2015}}]
  A subset $\Ideal \subseteq \Rx$ is an ideal if it satisfies:
  \begin{enumerate}
    \item $0 \in \Ideal$.
    \item If $f, g \in \Ideal$, then $f + g \in \Ideal$.
    \item If $f \in \Ideal$ and $h \in \Rx$, then $hf \in \Ideal$.
  \end{enumerate}
\end{definition}

\begin{definition}[{\thecite[Chapter~1, §4, Definition~2]{Cox-Little-Shea2015}}]
  Let $f_1, f_2, \ldots, f_m$ be polynomials in $\Rx$.
  Then we set
  \begin{align}
     \langle f_1, f_2, \ldots, f_m\rangle &= \Bigg\{\sum_{j=1}^mh_jf_j\ |\ h_1, h_2, \ldots, h_m\in\Rx\Bigg\}.
  \end{align}
\end{definition}

The set $\langle f_1, f_2, \ldots, f_m\rangle$ is an ideal and we call it the ideal generated by $f_1, f_2, \ldots, f_m$.
