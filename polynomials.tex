\chapter{Polynomials}
In this chapter, we provide all required background that is needed to understand how systems of polynomial equations can be efficiently solved and how to solve polynomial optimization problems.
A throughout description and understanding of this topic is required, since we will use the methods introduced in this chapter as building blocks later on in this thesis.

Firstly, we introduce basic notation of the polynomial algebra.
This will keep this chapter self-contained.

Secondly, the state-of-the-art methods for solving systems of polynomial equations over complex numbers will be reviewed.
First method is based on Gr\"obner bases computation, which is then used to create so-called multiplication matrix, which eigenvalues and eigenvectors are then used to compute the solutions.
The other method is numerical.
It is based on assumption that as we continuously transform an easy-to-solve polynomial system to the system we want to solve, the known solutions of the easy system are continuously transformed to the solutions we are looking for.

And thirdly, we will focus on polynomial optimization, i.e.\ optimizing a polynomial function with constraints given as polynomial equations and inequalities.
We will describe a method that finds the global optimum of a general polynomial optimization problem.
It is based on moments and on solving hierarchies of semidefinite problems.

\section{Polynomial algebra preliminaries}

\section{Algebraic geometry}

\section{Polynomial optimization}

\section{Conclusions}
