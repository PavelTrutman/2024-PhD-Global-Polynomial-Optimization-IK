\chapter{Polynomials}
In this chapter, we provide all required background that is needed to understand how systems of polynomial equations can be efficiently solved and how to solve polynomial optimization problems.
A throughout description and understanding of this topic is required, since we will use the methods introduced in this chapter as building blocks later on in this thesis.

Firstly, we introduce basic notation of the polynomial algebra.
This will keep this chapter self-contained.

Secondly, the state-of-the-art methods for solving systems of polynomial equations over complex numbers will be reviewed.
First method is based on Gr\"obner bases computation, which is then used to create so-called multiplication matrix, which eigenvalues and eigenvectors are then used to compute the solutions.
The other method is numerical.
It is based on assumption that as we continuously transform an easy-to-solve polynomial system to the system we want to solve, the known solutions of the easy system are continuously transformed to the solutions we are looking for.

And thirdly, we will focus on polynomial optimization, i.e.\ optimizing a polynomial function with constraints given as polynomial equations and inequalities.
We will describe a method that finds the global optimum of a general polynomial optimization problem.
It is based on moments and on solving hierarchies of semidefinite problems.

\section{Polynomial algebra preliminaries}
To make this chapter self-contained, we introduce the minimal possible amount of notation for polynomial algebra.
The overall review of the key concepts can be seen in \cite{Cox-Little-Shea2015}, which notation we will follow.

\subsection{The polynomial ring, ideals and varieties}
In this section, we will consider polynomials in $n$ unknowns $x_1, x_2, \ldots, x_n$ with real coefficients.

\begin{definition}[Monomial {\thecite[Chapter~1, §1, Definition~1]{Cox-Little-Shea2015}}]
  A monomial in variables $x_1, x_2, \ldots , x_n$ is a product of the form
  \begin{align}
    x_1^{\alpha_1} x_2^{\alpha_2} \cdots x_n^{\alpha_n},
  \end{align}
  where all of the exponents $\alpha_1, \alpha_2, \ldots, \alpha_n \in \N$.
  The total degree of this monomial is the sum $\alpha_1 + \alpha_2 + \cdots + \alpha_n$.
\end{definition}

When we let $\alpha = \begin{pmatrix}\alpha_1 & \alpha_2 & \cdots & \alpha_n\end{pmatrix}$ to be an $N$-tuple then we can simplify the notation for monomials by setting
\begin{align}
  x^\alpha &= x_1^{\alpha_1} x_2^{\alpha_2} \cdots x_n^{\alpha_n}.
\end{align}
We call the $N$-tuple $\alpha$ the multidegree of the monomial $x^\alpha$.
Let us also denote $|\alpha| = \alpha_1 + \alpha_2 + \cdots + \alpha_n$ the total degree of the monomial $x^\alpha$.
Note that $x^\alpha = 1$ when $\alpha = \begin{pmatrix}0 & 0 & \cdots & 0\end{pmatrix}$.

\begin{definition}[Polynomial {\thecite[Chapter~1, §1, Definition~2]{Cox-Little-Shea2015}}]
  A polynomial $f$ in variables $x_1, x_2, \cdots, x_n$ with real coefficients is a finite linear combination (with real coefficients) of monomials.
  We will write a polynomial $f$ in the form
  \begin{align}
    f &= \sum_{\alpha\in\N^n} c_\alpha x^\alpha,
  \end{align}
where $c_\alpha\in\R$, and there are only finitely many terms $c_\alpha x^\alpha$ in the sum.
\end{definition}

We call $c_\alpha$ the coefficient of the monomial $x^\alpha$.
If $c_\alpha \neq 0$, then we call $c_\alpha x^\alpha$ a term of the polynomial $f$.
The total degree of the polynomial $f$ is denoted as $\deg(f) = \max_{\alpha\in\N^n} |\alpha|$ for non-zero coefficients $c_\alpha\in\R$.
The zero polynomial has no degree.
The set of all polynomials in variables $x = \begin{bmatrix}x_1 & x_2 & \cdots & x_n\end{bmatrix}^\top$ with real coefficients is the ring of multivariate polynomials denoted as $\R[x]$.

\begin{example}
  Let us have polynomial $f \in\R[x, y, z]$ in three variables:
  \begin{align}
    f &= x^3y - 4xz^2 + \frac{2}{3}z^2 + 1.
  \end{align}
  The polynomial $f$ is a sum of four terms.
  The second term $-4xz^2$ with multidegree $\alpha = \begin{pmatrix}1 & 0 & 2\end{pmatrix}$ has a coefficient $-4$ and monomial $xz^2$.
  The total degree of this monomial is $1 + 0 + 2 = 3$, but the total degree $\deg(f)$ of the polynomial $f$ is 4, because of the monomial $x^3y$.
\end{example}


\section{Algebraic geometry}
%TODO ADD COX Using Algebraic Geometry

\section{Polynomial optimization}

\section{Conclusions}
